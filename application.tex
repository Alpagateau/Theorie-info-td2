\subsection{Compression d'images}
\subsubsection{Principe}

On se propose d'appliquer l'algorithme de Huffman en compressant des images en niveaux de gris, qui seront représentées, à l'aide de la bibliothèque Image de Python, par des tableaux d'entiers compris entre 0 et 255. 
On définit donc une fonction \texttt{encoder_decoder(image)} qui encode puis décode une image avec les fonctions définies précedemment, teste si l'image décodée est égale à l'originale, puis affiche à l'écran les valeurs de l'entropie empirique, de la longueur moyenne et du ratio de compression.
Pour une image donnée, on commence par afficher l'histogramme des valeurs prises, que l'on utilisera ensuite comme tableau de probabilités. On génère aussi un tableau d'entiers de 0 à 255 comme tableau des symboles. Pour respecter la précondition des fonctions de \texttt{huffman}, on doit trier préalablement le tableau des probabilités par ordre décroissant, en triant dans l'ordre correspondant le tableau de symboles.
Pour encoder l'image on joint le tableau des "mots" encodés en binaire en une seule chaîne de caractères, et pour la décoder, on utilise la fonction \texttt{decode} et la liste des mots de code correspondants aux symboles, générée grâce à la fonction \texttt{huffman_code}.
On affiche enfin à l'écran les valeurs demandées.

\subsubsection{Expérimentations}
