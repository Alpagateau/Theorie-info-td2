\documentclass[a4paper,11pt]{article}

%\usepackage[french]{babel}
\usepackage{color,xcolor,ucs}
\usepackage[top=1.2in, bottom=1.2in, left = 1in, right = 1in]{geometry}
\usepackage[linkcolor=black,colorlinks=true,urlcolor=blue]{hyperref}

%Includes
\usepackage{amsmath}
\usepackage{graphicx}
\usepackage{caption}
\usepackage{multicol}
\usepackage{tikz}
\usepackage{listings}

\setlength\parindent{0pt}
\lstset{
  basicstyle=\small,
  extendedchars=true,
  literate={à}{{\`a}}1 {<-}{{$\leftarrow$}}1,
}

\title{Rapport de travaux dirigés, Arbres de Huffman}
\author{TAMISIER Joshua - NADAUD Martin}

\begin{document}
\maketitle
\begin{center}
\rule{\textwidth}{1pt}

\end{center}

\vspace*{\fill}
\section{Introduction}

\noindent Dans ce TP, nous avons implémenté l'algorithme de Huffman dans le but d'encoder et de compresser des images ou du texte.
Pour ce faire, nous avons en premier lieu implémenté l'algorithme de compréssion en python, puis nous l'avons adapté aux diverses utilisations possible.

\newpage
%Table des matières
\tableofcontents
\newpage

\section{Algorithme de Huffman et codage source}
L'algorithme de Huffman est un algorithme de \textbf{codage source}. L'objectif est de réduire la longueur moyenne des mots de code. Voici le fonctionnement de l'algorithme de Huffman étape par étape.
\newline
Ici, nous gardon l'implémentation la plus générale possible. On ne sait pas quel type de données on encode, mais l'algorithme fonctionne de la même manière pour tout type.

\subsection{Arbre de Huffman}
La première étape est de construire un arbre binaire tel que la profondeur d'un mot\footnote{c'est a dire sa distance a la racine} dans l'arbre soit inversement correllée a sa fréquence d'apparition.\newline
Nous définissons un arbre de manière récursive en créant une classe \texttt{Noeud} avec deux enfants (droite et gauche) ainsi qu'une étiquette et un indice. L'algorithme utilisé fonctionne comme suit :
\newline
\begin{verbatim}
Entrée : p une Liste de nombre d'occurence de chaque mot de code.
Sortie : Arbre binaire de Huffman
Algorithme : 
arbres = liste_de_nombres_vers_liste_de_feuilles(p)
tant que arbres contient au moins 2 éléments
    trier(arbres)
    nouvel_arbre = Noeud(
      enfant_droit  : avant_dernier(arbres), 
      enfant_gauche : dernier(arbres)
    )
    arbres = concatener( tous_sauf_n_derniers(arbres, 2), nouvel_arbre)

retourner premier_élément(arbres)
\end{verbatim}

\begin{figure} [h]
    \centering
    \includegraphics[width=0.55 \linewidth]{huffman_alg.png}
    \caption*{Fonctionnement de l'algorithme de Huffman sur une liste d'entiers}
\end{figure}

Cet algorithme nous donne bien un arbre binaire, dont les feuilles correspondent aux éléments de la liste de départ, et dont la profondeur des feuilles est inversement correllée a leur valeur.

\subsection{Nouveau code}
De cet arbre, on peut définir une table de conversion \texttt{mot} $\rightarrow$ \texttt{code} de la façon suivante:
\begin{itemize}
  \item On crée une liste des mots encodées, on concatene 1 pour l'enfant de droite, 0 pour l'enfant de gauche.
  \item On associe chaque code a un mot source en fonction de sa fréquence. Mot très fréquent $\rightarrow$ Code court\footnote{On remarque qu'il n'existe pas une seule et unique façon d'associer les mots et les codes. En effet, deux codes de même longueur peuvent être interchangé sans impacter la validité de l'algorithme.}.
\end{itemize}

Grâce a cette table d'association, on peut alors encoder un à un les mots de la source, afin d'obtenir le message encodé.

\subsection{Décodage}
Une fois encoder, on peut décoder les données en utilisant le même arbre. Les codes de Huffman étant préfixes, le décode peut se faire de manière "gloutonne". Le décodage se fait en prenant chaque bit l'un après l'autre, en déçandant l'arbre jusqu'a arriver a une feuille. Par construction, on a la certitude que cette feuille correspond bien au mot encodé.
\newline
Ainsi, notre algorithme de décodage fonctionne ainsi :
\begin{verbatim}
  Entrée : seq une séquence de bits représentant les mots codés, 
           tree l'arbre utilisé pour les encoder.
  Sortie : une séquence de mots décodés
  Algorithme :
  offset := 0
  noeud_curseur := tree 
  décodé := []
  tant que seq n'est pas vide:
    si est_feuille(noeud_curseur) :
      seq = enlever_n_premier_bits(seq, longueure)
      décodé = concatener(décodé, mot(noeud_curseur))
      longueure = 0
      noeud_curseur = tree
    sinon
      si seq[0] = 0 : 
        noeud_curseur = noeud_curseur.enfant_gauche
      sinon : 
        noeud_curseur = neoud_curseur.enfant_droit
      longueur = longueur + 1

  retourner décodé
\end{verbatim}

\newpage
\section{Application}
\subsection{Compression d'images}
\subsubsection{Principe}

On se propose d'appliquer l'algorithme de Huffman en compressant des images en niveaux de gris, qui seront représentées, à l'aide de la bibliothèque Image de Python, par des tableaux d'entiers compris entre 0 et 255. 
\\
On définit donc une fonction \texttt{encoder\_decoder(image)} qui encode puis décode une image avec les fonctions définies précedemment, teste si l'image décodée est égale à l'originale, puis affiche à l'écran les valeurs de l'entropie empirique, de la longueur moyenne et du ratio de compression.
\\
Pour une image donnée, on commence par afficher l'histogramme des valeurs prises, que l'on utilisera ensuite comme tableau de probabilités. On génère aussi un tableau d'entiers de 0 à 255 comme tableau des symboles. Pour respecter la précondition des fonctions de \texttt{huffman}, on doit trier préalablement le tableau des probabilités par ordre décroissant, en triant dans l'ordre correspondant le tableau de symboles.
\\
Pour encoder l'image on joint le tableau des "mots" encodés en binaire en une seule chaîne de caractères, et pour la décoder, on utilise la fonction \texttt{decode} et la liste des mots de code correspondants aux symboles, générée grâce à la fonction \texttt{huffman\_code}.
On affiche enfin à l'écran les valeurs demandées.

\subsubsection{Expérimentations}

Ci-après figurent les images fournies ainsi que leurs histogrammes respectifs:

\begin{figure}[H]
	\centering
	\begin{subfigure}[b]{0.4\linewidth}
		\includegraphics[width=\linewidth]{goldhill.png}
		\caption{\texttt{goldhill.png}}
	\end{subfigure}
	\begin{subfigure}[b]{0.4\linewidth}
		\includegraphics[width=\linewidth]{goldhill_hist.png}
		\caption{Histogramme de \texttt{goldhill.png}}
	\end{subfigure}
	\label{fig:goldhill}
\end{figure}
\begin{figure}[H]
	\centering
	\begin{subfigure}[b]{0.4\linewidth}
		\includegraphics[width=\linewidth]{moon.png}
		\caption{\texttt{moon.png}}
	\end{subfigure}
	\begin{subfigure}[b]{0.4\linewidth}
		\includegraphics[width=\linewidth]{moon_hist.png}
		\caption{Histogramme de \texttt{moon.png}}
	\end{subfigure}
	\label{fig:moon}
\end{figure}
\begin{figure}[H]
	\centering
	\begin{subfigure}[b]{0.4\linewidth}
		\includegraphics[width=\linewidth]{boat.png}
		\caption{\texttt{boat.png}}
	\end{subfigure}
	\begin{subfigure}[b]{0.4\linewidth}
		\includegraphics[width=\linewidth]{boat_hist.png}
		\caption{Histogramme de \texttt{boat.png}}
	\end{subfigure}
	\label{fig:boat}
\end{figure}

Les valeurs expérimentales:

\begin{verbatim}

#goldhill.png

Entropie empirique: 7.472173225084263
Longueur moyenne du code: 7.526447610294115
Ratio de compression: 0.9371309280395508

=============================================

#moon.png

Entropie empirique: 2.465213838156748
Longueur moyenne du code: 2.714325138504536
Ratio de compression: 0.3379652882415313

=============================================

#boat.png

Entropie empirique: 7.185723654928107
Longueur moyenne du code: 7.246993719362746
Ratio de compression: 0.9023356437683105

\end{verbatim}

On remarque deux choses:
\begin{itemize}
	\item Les longueurs moyennes sont bien comprises entre H et H+1, donc le code respecte bien le théorème de Shannon.
	\item Le ratio de compression et la longueur moyenne associés à \texttt{moon.png} sont largement inférieurs à ceux des deux autres images, ce qui s'accorde avec le résultat attendu, car comme on peut le voir sur l'image et sur son histogramme, elle est composée principalement de pixels noirs, donc l'entropie de l'image est moindre (elle porte peu d'information) et il est possible de la compresser davantage.
\end{itemize}

\subsection{Compression de texte}
\subsubsection{Principe}
Pour la compression de texte, le principe général est très similaire, a quelques différences près. Au lieu d'utiliser comme alphabet source les entiers de 0 a 255, on utilise ici la table ascii (donc des entiers de 0 a 127) réduite aux charactères presents dans le texte. \\
Aussi, en lieu d'histogramme, on compte le nombre d'occurences de chaques charactères manuellement pour en obtenir la fréquence. Ce sont ensuite les mêmes fonctions qui permettent d'encoder et de décoder le texte.\\
On définit donc une fonction \texttt{encoder\_décoder\_texte()} qui comme son nom l'indique, encode puis décode un texte, en comparant que le texte décodé est égale a l'entrée. Cette fonctionne retourne aussi l'entropie empirique du texte donné, la longueur moyenne des mots encodés et le taux de compréssion.

\subsubsection{Experimentation}

Voici les resultats des la compression de textes : \newline
\begin{center}
\begin{tabular}{|l|c|c|c|}
  \hline texte source & entropie & longueur moyenne & ratio de compression \\
  \hline
    buscon & 4.41 & 4.45 & 0.55 \\
    candide & 4.57 & 4.59 & 0.57 \\
    dorian & 4.48 & 4.52 & 0.56 \\
    clair de lune & 4.50 & 4.53 & 0.56 \\
    \hline
\end{tabular}
\end{center}

On remarque que la taille finale des textes est environs réduite de moitié. On peut attribuer ça a la loi de Zipf qui stipule que dans une langue naturelle, le charactère le plus présent est environs deux fois plus commun que le second plus présent (et ainsi de suite). On peut alors supposer que quelques charactères sont représenté par des mots très cours, mais que ceux-ci représente une grosse partie du texte source.

\newpage
\section{Pour aller plus loins}
\section{Pour aller plus loin}
La question est, peut-on améliorer cet algorithme? On sait que les codes de Huffman sont optimaux, et donc le seul moyen d'améliorer le ratio de compression serait de réduire l'entropie de la source. Voici donc une proposition d'algorithme pour réduire l'entropie de la source avant de la compresser par Huffman.

\subsection{Proposition d'algorithme}
L'idée de cet algorithme consiste a remplacer la valeur de chaque pixel par la différence entre lui et le pixel précédent. C'est ce qu'on appelle du \texttt{Delta Encoding}.
Voici un exemple de code C pour encoder un tableau ainsi :

\begin{verbatim}
void encode(static char* input, int input_len, char* output)
{
  //On suppose que output est initialisé a 0
  output[0] = input[0];
  for(int i = 1; i < input_len; i++)
  {
    output[i] = input[i] - input[i-1];
  }
}
\end{verbatim}

Si une image n'est pas aléatoire, c'est-à-dire qu'elle contient des patterns (comme par exemple une photographie), on peut s'attendre à des différences similaires à plusieurs endroits de la photo. Par exemple, avec cet encodage préalable, tous les aplats de couleurs seront représentés par des 0. On peut donc s'attendre à une entropie réduite, et donc à un code de Huffman plus efficace. Cette méthode est notamment utilisée dans les images aux format \texttt{png} \footnote{https://www.w3.org/TR/png-3/\#7Filtering}.


\end{document}
